Since the beginning of the GNSS technologies, the accuracy has improved to around 1-meter precision\cite{galileoHasPolicy}. The first step to improve the accuracy for the general public was to remove the GPS selective availability in May 2000\cite{galileoHasPolicy}. Afterwards, the accuracy was improved by reducing the error in orbits calculation, ionospheric propagation models and clocks\cite{galileoHasPolicy}. Another major improvement was the development of satellite-based augmentation systems (SBAS). These augmentation systems were developed for the requirements of civil aviation and they compute positioning integrity and transmit satellite and ionospheric corrections\cite{galileoHasPolicy}. Examples of these systems are the American WAAS and the European EGNOS. These systems however, only have regional coverage area.

To further improve the accuracy, Differential GNSS (DGNSS) and Real-time Kinematic (RTK) techniques have been developed. Differential GNSS works by measuring the most of errors in a nearby station\cite{galileoHasPolicy}. For sub-decimeter-level positioning accuracy, it is required to use carrier phase measurements. RTK is similar to DGNSS, but applied to carrier phase \cite{galileoHasPolicy}. The differential GNSS techniques requires a second nearby station with a known position, therefore it is limited to an area of tens of kilometers\cite{galileoHasPolicy}.

On the other hand, Precise Point Positioning(PPP) provides centimeter-level accuracy and requires only one receiver. This technique involves a stream of absolute corrections to the GNSS Receiver. Ionospheric errors can be corrected by using the propagation of multiple signals on different frequencies. PPP techniques are still under development. Initial methods converged to a solution in hours, while recent developments have much faster convergence time\cite{galileoHasPolicy}\cite{instantPPP}. 