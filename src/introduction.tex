% vim: set tw=78 sts=2 sw=2 ts=8 aw et ai:

Global Navigation Satellite System (GNSS) positioning relies on computing the pseudoranges between the receiver and the satellite. The pseudoranges are calculated by measuring the transmitting time of the signals emitted by the satellites and multiplying with the speed of light. Ranges to four satellites and their orbital positions are required for computing the positioning solution. Since there are many possible inaccuracies in the measuring chain of the propagation time, we use the term pseudoranges for these distances to the satellites.

The first GNSS receivers consisted of large analog equipment, back in the 1970s when GNSS was a military-only technology\cite{receivers}. Now, due to their popularity and wide area of applications, they are mostly miniaturized platforms implemented on microprocessors, Integrated Circuits (IC), FPGAs, DSPs\cite{receivers}. They are found in almost all mobile phones. Usually, GNSS receivers are implemented in hardware, that means it is a dedicated chip designed specifically for this purpose. However, with the development of software defined radio (SDR) technologies, software GNSS receiver becoming increasingly popular because they have the advantage of increased flexibility.

The European Commission decided to offer the Galileo High Accuracy signal for free in order to allow the industry and the community to innovate both emerging and estabilished markets\cite{galileoHasPolicy}. The Galileo Commercial Service, which aims to provide precise point positioning will broadcast corrections over the E6B signal while The E6C signal will be used for authentication\cite{e6breceiver}. Most aspects of the signals are public and can be found in the Galileo Signal in Space Interface Control Document \cite{galileoSisIcd}.

In this work, we will present the state-of-the-art of GNSS accuracy and high accuracy positioning methods, then we look at the Galileo HAS and Galileo E6 signal and finally, we review some papers presenting GNSS receiver implementations.
