% vim: set tw=78 sts=2 sw=2 ts=8 aw et ai:

First developed in the 20th century as a military technology, Global Navigation Satellite Systems (GNSS) soon found numerous civilian applications. Nowadays, GNSS has become a part of our everyday life and it is an enabling technology in many areas, from consumer electronics, to aviation, transportation, agriculture and geomatics. Initially, the system offered an accuracy of about five meters, but algorithms and technologies for improving the accuracy have been developed and current GNSS receivers provide a precision of less than 30 centimeters with professional-grade equipment being able of sub-milimeter accuracy. With the emergence of autonomous cars, unmanned vehicles and robotics, the GNSS market is only expected to grow even further. Galileo, the European Union civilian GNSS, aims to provide higher-precision capabilities for free. In this work we will explore current state-of-the-art solutions for GNSS signals decoding based on Software Defined Radio, DSP and FPGA and corrections applications for high-accuracy positioning.

\textbf{Keywords:} GNSS, Galileo, PPP, SDR, FPGA, DSP